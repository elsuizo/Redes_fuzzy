
% Default to the notebook output style

    


% Inherit from the specified cell style.




    
\documentclass{article}

    
    
    \usepackage{graphicx} % Used to insert images
   % \usepackage{adjustbox} % Used to constrain images to a maximum size 
    \usepackage{color} % Allow colors to be defined
    \usepackage{enumerate} % Needed for markdown enumerations to work
    \usepackage{geometry} % Used to adjust the document margins
    \usepackage{amsmath} % Equations
    \usepackage{amssymb} % Equations
    \usepackage[mathletters]{ucs} % Extended unicode (utf-8) support
    \usepackage[utf8x]{inputenc} % Allow utf-8 characters in the tex document
    \usepackage{fancyvrb} % verbatim replacement that allows latex
    \usepackage{grffile} % extends the file name processing of package graphics 
                         % to support a larger range 
    % The hyperref package gives us a pdf with properly built
    % internal navigation ('pdf bookmarks' for the table of contents,
    % internal cross-reference links, web links for URLs, etc.)
    \usepackage{hyperref}
    \usepackage{longtable} % longtable support required by pandoc >1.10
    

    
    
    \definecolor{orange}{cmyk}{0,0.4,0.8,0.2}
    \definecolor{darkorange}{rgb}{.71,0.21,0.01}
    \definecolor{darkgreen}{rgb}{.12,.54,.11}
    \definecolor{myteal}{rgb}{.26, .44, .56}
    \definecolor{gray}{gray}{0.45}
    \definecolor{lightgray}{gray}{.95}
    \definecolor{mediumgray}{gray}{.8}
    \definecolor{inputbackground}{rgb}{.95, .95, .85}
    \definecolor{outputbackground}{rgb}{.95, .95, .95}
    \definecolor{traceback}{rgb}{1, .95, .95}
    % ansi colors
    \definecolor{red}{rgb}{.6,0,0}
    \definecolor{green}{rgb}{0,.65,0}
    \definecolor{brown}{rgb}{0.6,0.6,0}
    \definecolor{blue}{rgb}{0,.145,.698}
    \definecolor{purple}{rgb}{.698,.145,.698}
    \definecolor{cyan}{rgb}{0,.698,.698}
    \definecolor{lightgray}{gray}{0.5}
    
    % bright ansi colors
    \definecolor{darkgray}{gray}{0.25}
    \definecolor{lightred}{rgb}{1.0,0.39,0.28}
    \definecolor{lightgreen}{rgb}{0.48,0.99,0.0}
    \definecolor{lightblue}{rgb}{0.53,0.81,0.92}
    \definecolor{lightpurple}{rgb}{0.87,0.63,0.87}
    \definecolor{lightcyan}{rgb}{0.5,1.0,0.83}
    
    % commands and environments needed by pandoc snippets
    % extracted from the output of `pandoc -s`
    \DefineVerbatimEnvironment{Highlighting}{Verbatim}{commandchars=\\\{\}}
    % Add ',fontsize=\small' for more characters per line
    \newenvironment{Shaded}{}{}
    \newcommand{\KeywordTok}[1]{\textcolor[rgb]{0.00,0.44,0.13}{\textbf{{#1}}}}
    \newcommand{\DataTypeTok}[1]{\textcolor[rgb]{0.56,0.13,0.00}{{#1}}}
    \newcommand{\DecValTok}[1]{\textcolor[rgb]{0.25,0.63,0.44}{{#1}}}
    \newcommand{\BaseNTok}[1]{\textcolor[rgb]{0.25,0.63,0.44}{{#1}}}
    \newcommand{\FloatTok}[1]{\textcolor[rgb]{0.25,0.63,0.44}{{#1}}}
    \newcommand{\CharTok}[1]{\textcolor[rgb]{0.25,0.44,0.63}{{#1}}}
    \newcommand{\StringTok}[1]{\textcolor[rgb]{0.25,0.44,0.63}{{#1}}}
    \newcommand{\CommentTok}[1]{\textcolor[rgb]{0.38,0.63,0.69}{\textit{{#1}}}}
    \newcommand{\OtherTok}[1]{\textcolor[rgb]{0.00,0.44,0.13}{{#1}}}
    \newcommand{\AlertTok}[1]{\textcolor[rgb]{1.00,0.00,0.00}{\textbf{{#1}}}}
    \newcommand{\FunctionTok}[1]{\textcolor[rgb]{0.02,0.16,0.49}{{#1}}}
    \newcommand{\RegionMarkerTok}[1]{{#1}}
    \newcommand{\ErrorTok}[1]{\textcolor[rgb]{1.00,0.00,0.00}{\textbf{{#1}}}}
    \newcommand{\NormalTok}[1]{{#1}}
    
    % Define a nice break command that doesn't care if a line doesn't already
    % exist.
    \def\br{\hspace*{\fill} \\* }
    % Math Jax compatability definitions
    \def\gt{>}
    \def\lt{<}
    % Document parameters
    \title{guia1}
    
    
    

    % Pygments definitions
    
\makeatletter
\def\PY@reset{\let\PY@it=\relax \let\PY@bf=\relax%
    \let\PY@ul=\relax \let\PY@tc=\relax%
    \let\PY@bc=\relax \let\PY@ff=\relax}
\def\PY@tok#1{\csname PY@tok@#1\endcsname}
\def\PY@toks#1+{\ifx\relax#1\empty\else%
    \PY@tok{#1}\expandafter\PY@toks\fi}
\def\PY@do#1{\PY@bc{\PY@tc{\PY@ul{%
    \PY@it{\PY@bf{\PY@ff{#1}}}}}}}
\def\PY#1#2{\PY@reset\PY@toks#1+\relax+\PY@do{#2}}

\def\PY@tok@gd{\def\PY@tc##1{\textcolor[rgb]{0.63,0.00,0.00}{##1}}}
\def\PY@tok@gu{\let\PY@bf=\textbf\def\PY@tc##1{\textcolor[rgb]{0.50,0.00,0.50}{##1}}}
\def\PY@tok@gt{\def\PY@tc##1{\textcolor[rgb]{0.00,0.25,0.82}{##1}}}
\def\PY@tok@gs{\let\PY@bf=\textbf}
\def\PY@tok@gr{\def\PY@tc##1{\textcolor[rgb]{1.00,0.00,0.00}{##1}}}
\def\PY@tok@cm{\let\PY@it=\textit\def\PY@tc##1{\textcolor[rgb]{0.25,0.50,0.50}{##1}}}
\def\PY@tok@vg{\def\PY@tc##1{\textcolor[rgb]{0.10,0.09,0.49}{##1}}}
\def\PY@tok@m{\def\PY@tc##1{\textcolor[rgb]{0.40,0.40,0.40}{##1}}}
\def\PY@tok@mh{\def\PY@tc##1{\textcolor[rgb]{0.40,0.40,0.40}{##1}}}
\def\PY@tok@go{\def\PY@tc##1{\textcolor[rgb]{0.50,0.50,0.50}{##1}}}
\def\PY@tok@ge{\let\PY@it=\textit}
\def\PY@tok@vc{\def\PY@tc##1{\textcolor[rgb]{0.10,0.09,0.49}{##1}}}
\def\PY@tok@il{\def\PY@tc##1{\textcolor[rgb]{0.40,0.40,0.40}{##1}}}
\def\PY@tok@cs{\let\PY@it=\textit\def\PY@tc##1{\textcolor[rgb]{0.25,0.50,0.50}{##1}}}
\def\PY@tok@cp{\def\PY@tc##1{\textcolor[rgb]{0.74,0.48,0.00}{##1}}}
\def\PY@tok@gi{\def\PY@tc##1{\textcolor[rgb]{0.00,0.63,0.00}{##1}}}
\def\PY@tok@gh{\let\PY@bf=\textbf\def\PY@tc##1{\textcolor[rgb]{0.00,0.00,0.50}{##1}}}
\def\PY@tok@ni{\let\PY@bf=\textbf\def\PY@tc##1{\textcolor[rgb]{0.60,0.60,0.60}{##1}}}
\def\PY@tok@nl{\def\PY@tc##1{\textcolor[rgb]{0.63,0.63,0.00}{##1}}}
\def\PY@tok@nn{\let\PY@bf=\textbf\def\PY@tc##1{\textcolor[rgb]{0.00,0.00,1.00}{##1}}}
\def\PY@tok@no{\def\PY@tc##1{\textcolor[rgb]{0.53,0.00,0.00}{##1}}}
\def\PY@tok@na{\def\PY@tc##1{\textcolor[rgb]{0.49,0.56,0.16}{##1}}}
\def\PY@tok@nb{\def\PY@tc##1{\textcolor[rgb]{0.00,0.50,0.00}{##1}}}
\def\PY@tok@nc{\let\PY@bf=\textbf\def\PY@tc##1{\textcolor[rgb]{0.00,0.00,1.00}{##1}}}
\def\PY@tok@nd{\def\PY@tc##1{\textcolor[rgb]{0.67,0.13,1.00}{##1}}}
\def\PY@tok@ne{\let\PY@bf=\textbf\def\PY@tc##1{\textcolor[rgb]{0.82,0.25,0.23}{##1}}}
\def\PY@tok@nf{\def\PY@tc##1{\textcolor[rgb]{0.00,0.00,1.00}{##1}}}
\def\PY@tok@si{\let\PY@bf=\textbf\def\PY@tc##1{\textcolor[rgb]{0.73,0.40,0.53}{##1}}}
\def\PY@tok@s2{\def\PY@tc##1{\textcolor[rgb]{0.73,0.13,0.13}{##1}}}
\def\PY@tok@vi{\def\PY@tc##1{\textcolor[rgb]{0.10,0.09,0.49}{##1}}}
\def\PY@tok@nt{\let\PY@bf=\textbf\def\PY@tc##1{\textcolor[rgb]{0.00,0.50,0.00}{##1}}}
\def\PY@tok@nv{\def\PY@tc##1{\textcolor[rgb]{0.10,0.09,0.49}{##1}}}
\def\PY@tok@s1{\def\PY@tc##1{\textcolor[rgb]{0.73,0.13,0.13}{##1}}}
\def\PY@tok@sh{\def\PY@tc##1{\textcolor[rgb]{0.73,0.13,0.13}{##1}}}
\def\PY@tok@sc{\def\PY@tc##1{\textcolor[rgb]{0.73,0.13,0.13}{##1}}}
\def\PY@tok@sx{\def\PY@tc##1{\textcolor[rgb]{0.00,0.50,0.00}{##1}}}
\def\PY@tok@bp{\def\PY@tc##1{\textcolor[rgb]{0.00,0.50,0.00}{##1}}}
\def\PY@tok@c1{\let\PY@it=\textit\def\PY@tc##1{\textcolor[rgb]{0.25,0.50,0.50}{##1}}}
\def\PY@tok@kc{\let\PY@bf=\textbf\def\PY@tc##1{\textcolor[rgb]{0.00,0.50,0.00}{##1}}}
\def\PY@tok@c{\let\PY@it=\textit\def\PY@tc##1{\textcolor[rgb]{0.25,0.50,0.50}{##1}}}
\def\PY@tok@mf{\def\PY@tc##1{\textcolor[rgb]{0.40,0.40,0.40}{##1}}}
\def\PY@tok@err{\def\PY@bc##1{\fcolorbox[rgb]{1.00,0.00,0.00}{1,1,1}{##1}}}
\def\PY@tok@kd{\let\PY@bf=\textbf\def\PY@tc##1{\textcolor[rgb]{0.00,0.50,0.00}{##1}}}
\def\PY@tok@ss{\def\PY@tc##1{\textcolor[rgb]{0.10,0.09,0.49}{##1}}}
\def\PY@tok@sr{\def\PY@tc##1{\textcolor[rgb]{0.73,0.40,0.53}{##1}}}
\def\PY@tok@mo{\def\PY@tc##1{\textcolor[rgb]{0.40,0.40,0.40}{##1}}}
\def\PY@tok@kn{\let\PY@bf=\textbf\def\PY@tc##1{\textcolor[rgb]{0.00,0.50,0.00}{##1}}}
\def\PY@tok@mi{\def\PY@tc##1{\textcolor[rgb]{0.40,0.40,0.40}{##1}}}
\def\PY@tok@gp{\let\PY@bf=\textbf\def\PY@tc##1{\textcolor[rgb]{0.00,0.00,0.50}{##1}}}
\def\PY@tok@o{\def\PY@tc##1{\textcolor[rgb]{0.40,0.40,0.40}{##1}}}
\def\PY@tok@kr{\let\PY@bf=\textbf\def\PY@tc##1{\textcolor[rgb]{0.00,0.50,0.00}{##1}}}
\def\PY@tok@s{\def\PY@tc##1{\textcolor[rgb]{0.73,0.13,0.13}{##1}}}
\def\PY@tok@kp{\def\PY@tc##1{\textcolor[rgb]{0.00,0.50,0.00}{##1}}}
\def\PY@tok@w{\def\PY@tc##1{\textcolor[rgb]{0.73,0.73,0.73}{##1}}}
\def\PY@tok@kt{\def\PY@tc##1{\textcolor[rgb]{0.69,0.00,0.25}{##1}}}
\def\PY@tok@ow{\let\PY@bf=\textbf\def\PY@tc##1{\textcolor[rgb]{0.67,0.13,1.00}{##1}}}
\def\PY@tok@sb{\def\PY@tc##1{\textcolor[rgb]{0.73,0.13,0.13}{##1}}}
\def\PY@tok@k{\let\PY@bf=\textbf\def\PY@tc##1{\textcolor[rgb]{0.00,0.50,0.00}{##1}}}
\def\PY@tok@se{\let\PY@bf=\textbf\def\PY@tc##1{\textcolor[rgb]{0.73,0.40,0.13}{##1}}}
\def\PY@tok@sd{\let\PY@it=\textit\def\PY@tc##1{\textcolor[rgb]{0.73,0.13,0.13}{##1}}}

\def\PYZbs{\char`\\}
\def\PYZus{\char`\_}
\def\PYZob{\char`\{}
\def\PYZcb{\char`\}}
\def\PYZca{\char`\^}
\def\PYZsh{\char`\#}
\def\PYZpc{\char`\%}
\def\PYZdl{\char`\$}
\def\PYZti{\char`\~}
% for compatibility with earlier versions
\def\PYZat{@}
\def\PYZlb{[}
\def\PYZrb{]}
\makeatother


    % Exact colors from NB
    \definecolor{incolor}{rgb}{0.0, 0.0, 0.5}
    \definecolor{outcolor}{rgb}{0.545, 0.0, 0.0}



    
    % Prevent overflowing lines due to hard-to-break entities
    \sloppy 
    % Setup hyperref package
    \hypersetup{
      breaklinks=true,  % so long urls are correctly broken across lines
      colorlinks=true,
      urlcolor=blue,
      linkcolor=darkorange,
      citecolor=darkgreen,
      }
    % Slightly bigger margins than the latex defaults
    
    \geometry{verbose,tmargin=1in,bmargin=1in,lmargin=1in,rmargin=1in}
    
    \title{\textbf{Redes Neuronales y Control Difuso} \\ \vspace{1cm} Guia1 \\Martin Noblía \\ Marcos Panizzo \\ Damián Presti}

    \begin{document}
    
    
    \maketitle
    
   \begin{figure}[h!]
   \centering
   \includegraphics[width=0.23\textwidth]{logo_iaci.eps}
   \end{figure}

    
%\section{Redes Neuronales y Control difuso}
%
%\subsection{Guia 1}
%
%\begin{center}
%
%\Huge{Martin Noblía \\ Marcos Panizzo \\ Damián Presti}
%
%\end{center}
%
%
%
%    Guia1 por Martin Noblía Marcos Panizzo Damián Presti se distribuye bajo
%una Licencia Creative Commons Atribución-CompartirIgual 4.0
%Internacional.

\section{Ejercicio 1}

    En este ejercicio trabajaremos con una neurona I\&F con los siguientes
parámetros:

\begin{Shaded}
\begin{Highlighting}[]
 \NormalTok{tau = }\DecValTok{20}\NormalTok{.  }\CommentTok{# time constant(ms)}
 \NormalTok{R = }\DecValTok{10}\NormalTok{.  }\CommentTok{# input Resistance(MOhms)}
 \NormalTok{C = tau / R  }\CommentTok{# capacidad}
 \NormalTok{v_thres = }\DecValTok{40}  \CommentTok{# thresold voltage(mV)}
\end{Highlighting}
\end{Shaded}
\begin{enumerate}[a)]
\item
  Programar la Neurona cuando se aplica un pulso de corrient constante (
  $I_{max}$ )
\item
  Usar el método de Euler forward y backward. Mostrar que el método
  forward se vuelve inestable si $dt > 2 \tau$
\item
  Calcular la curva teórica que predice la frecuencia en función de la
  corriente aplicada.
\item
  Verificar la predicción por medio de simulaciones. O sea, simular
  distintos valores de corrientes y medir la frecuencia de emisión de
  spikes. Solapar sus mediciones computacionales con la curva teórica en
  c).
\end{enumerate}

    \subsubsection{a)}

\subsection{Neuronas Integrate and Fire:}

En este modelo suponemos que debajo de un potencial umbral $v_{thres}$
de disparo la neurona se comporta como un circuito RC y cuando sobrepasa
$v_{thres}$ dispara un spike.

Como sabemos la ecuación diferencial que modela un circuito RC es:

\[C\frac{dv}{dt}=\frac{-v}{R}+I\]

o equivalentemente:

\[\frac{dv}{dt}=\frac{-v}{RC}+\frac{I}{C}\]

Para resolver numéricamente esta ecuación diferencial lineal se nos
proponen los métodos de Euler(\emph{Forward} y \emph{Backward}) que
consisten en aproximar la derivada como $\frac{V^{j}-V^{j-1}}{dt}$,
donde $dt$ es el paso de discretización y en el caso \emph{forward} la
iteración de referencia es la $(j-1)$ por ello la ecuación diferencial
discretizada queda:

\[\frac{V^{j}-V^{j-1}}{dt} + \frac{V^{j-1}}{\tau}=f^{j-1}\]

o equivalentemente:

\[V^{j}=(1-\frac{dt}{\tau}) V^{j-1}+dt f^{j-1}\]

Para el caso del método \emph{Backward} la iteración de referencia es la
$(j)$ por ello la ecuación diferencial discretizada queda:

\[\frac{V^{j}-V^{j-1}}{dt} + \frac{V^{j}}{\tau}=f^{j}\]

o equivalentemente:

\[V^{j}=\frac{V^{j-1}+dt f^{j}}{1+\frac{dt}{\tau}}\]

Modelamos a la neurona de acuerdo a la siguiente clase que tiene un
metodo de simulación pedido

    \begin{Verbatim}[commandchars=\\\{\}]
{\color{incolor}In [{\color{incolor}29}]:} \PY{k+kn}{import} \PY{n+nn}{numpy} \PY{k+kn}{as} \PY{n+nn}{np}
         
         \PY{k}{class} \PY{n+nc}{NeuronaIF}\PY{p}{:}
             \PY{l+s+sd}{"""}
         \PY{l+s+sd}{    Clase NeuronaIF que modela una neurona Integrate and fire}
         \PY{l+s+sd}{    }
         \PY{l+s+sd}{    """}
             
             \PY{c}{\PYZsh{}Constructor de la clase}
             \PY{k}{def} \PY{n+nf}{\PYZus{}\PYZus{}init\PYZus{}\PYZus{}}\PY{p}{(}\PY{n+nb+bp}{self}\PY{p}{,} \PY{n}{tau}\PY{p}{,} \PY{n}{R}\PY{p}{,} \PY{n}{v\PYZus{}thres}\PY{p}{)}\PY{p}{:}
                 \PY{n+nb+bp}{self}\PY{o}{.}\PY{n}{tau} \PY{o}{=} \PY{n}{tau} 
                 \PY{n+nb+bp}{self}\PY{o}{.}\PY{n}{R} \PY{o}{=} \PY{n}{R}
                 \PY{n+nb+bp}{self}\PY{o}{.}\PY{n}{v\PYZus{}thres} \PY{o}{=} \PY{n}{v\PYZus{}thres}
                 \PY{n+nb+bp}{self}\PY{o}{.}\PY{n}{C} \PY{o}{=} \PY{n}{tau} \PY{o}{/} \PY{n}{R}
                 
             \PY{c}{\PYZsh{} Metodo de simulacion}
             \PY{k}{def} \PY{n+nf}{simulate}\PY{p}{(}\PY{n+nb+bp}{self}\PY{p}{,} \PY{n}{dt}\PY{o}{=}\PY{l+m+mi}{1}\PY{p}{,} \PY{n}{method}\PY{o}{=}\PY{l+s}{'}\PY{l+s}{forward}\PY{l+s}{'}\PY{p}{,} \PY{n}{t\PYZus{}max}\PY{o}{=}\PY{l+m+mi}{1000}\PY{p}{,} \PY{n}{I\PYZus{}max}\PY{o}{=}\PY{o}{.}\PY{l+m+mi}{2}\PY{p}{,} \PY{n}{n\PYZus{}sims}\PY{o}{=}\PY{l+m+mi}{1}\PY{p}{)}\PY{p}{:}
                 \PY{l+s+sd}{"""Metodo para simular la neurona"""}
                 \PY{n+nb+bp}{self}\PY{o}{.}\PY{n}{n\PYZus{}sims} \PY{o}{=} \PY{n}{n\PYZus{}sims}
                 \PY{n+nb+bp}{self}\PY{o}{.}\PY{n}{I\PYZus{}max} \PY{o}{=} \PY{n}{I\PYZus{}max}
                 \PY{n+nb+bp}{self}\PY{o}{.}\PY{n}{dt} \PY{o}{=} \PY{n}{dt}
                 \PY{n+nb+bp}{self}\PY{o}{.}\PY{n}{t\PYZus{}max} \PY{o}{=} \PY{n}{t\PYZus{}max}
                 \PY{n+nb+bp}{self}\PY{o}{.}\PY{n}{method} \PY{o}{=} \PY{n}{method}
                 
                 \PY{n}{t\PYZus{}v} \PY{o}{=} \PY{n}{np}\PY{o}{.}\PY{n}{arange}\PY{p}{(}\PY{l+m+mi}{0}\PY{p}{,}\PY{n+nb+bp}{self}\PY{o}{.}\PY{n}{t\PYZus{}max} \PY{o}{-} \PY{n+nb+bp}{self}\PY{o}{.}\PY{n}{dt} \PY{o}{+} \PY{l+m+mi}{1}\PY{p}{,} \PY{n}{dt}\PY{p}{,}\PY{n}{dtype}\PY{o}{=}\PY{n+nb}{float}\PY{p}{)}
                 \PY{n}{nt\PYZus{}v} \PY{o}{=} \PY{n+nb}{len}\PY{p}{(}\PY{n}{t\PYZus{}v}\PY{p}{)}
                 
                 \PY{n}{dt\PYZus{}tau} \PY{o}{=} \PY{n+nb+bp}{self}\PY{o}{.}\PY{n}{dt} \PY{o}{/} \PY{n+nb+bp}{self}\PY{o}{.}\PY{n}{tau}
                 \PY{n}{one\PYZus{}dt\PYZus{}tau} \PY{o}{=} \PY{l+m+mi}{1} \PY{o}{-} \PY{n}{dt\PYZus{}tau} \PY{c}{\PYZsh{} forward Euler}
                 \PY{n}{m1\PYZus{}dt\PYZus{}tau} \PY{o}{=} \PY{l+m+mi}{1} \PY{o}{/} \PY{p}{(}\PY{l+m+mi}{1} \PY{o}{+} \PY{n}{dt\PYZus{}tau}\PY{p}{)} \PY{c}{\PYZsh{} backward Euler}
                 \PY{n}{Nt} \PY{o}{=} \PY{n+nb}{len}\PY{p}{(}\PY{n}{t\PYZus{}v}\PY{p}{)}
                 \PY{n}{i\PYZus{}v\PYZus{}i} \PY{o}{=} \PY{n}{np}\PY{o}{.}\PY{n}{zeros\PYZus{}like}\PY{p}{(}\PY{n}{t\PYZus{}v}\PY{p}{)} \PY{c}{\PYZsh{} nA}
         
                 \PY{c}{\PYZsh{}i\PYZus{}v\PYZus{}i[(Nt*3/4.):(Nt*3/4.))}
                 \PY{n}{i\PYZus{}v\PYZus{}i}\PY{p}{[}\PY{l+m+mi}{0}\PY{p}{:}\PY{p}{(}\PY{n}{Nt}\PY{o}{*}\PY{l+m+mi}{3}\PY{o}{/}\PY{l+m+mi}{4}\PY{p}{)}\PY{p}{]} \PY{o}{=} \PY{n}{I\PYZus{}max} \PY{c}{\PYZsh{} nA}
                 
                 \PY{k}{for} \PY{n}{j} \PY{o+ow}{in} \PY{n+nb}{xrange}\PY{p}{(}\PY{l+m+mi}{0}\PY{p}{,} \PY{n+nb+bp}{self}\PY{o}{.}\PY{n}{n\PYZus{}sims}\PY{p}{)}\PY{p}{:}
                     
                     
                     \PY{n}{i\PYZus{}v\PYZus{}tot} \PY{o}{=} \PY{n}{i\PYZus{}v\PYZus{}i}
         
                     \PY{n}{v\PYZus{}v} \PY{o}{=} \PY{n}{np}\PY{o}{.}\PY{n}{zeros\PYZus{}like}\PY{p}{(}\PY{n}{t\PYZus{}v}\PY{p}{)}
                     \PY{n}{s\PYZus{}v} \PY{o}{=} \PY{n}{np}\PY{o}{.}\PY{n}{zeros\PYZus{}like}\PY{p}{(}\PY{n}{t\PYZus{}v}\PY{p}{)}
         
                     \PY{k}{for} \PY{n}{i} \PY{o+ow}{in} \PY{n+nb}{xrange}\PY{p}{(}\PY{l+m+mi}{1}\PY{p}{,}\PY{n}{nt\PYZus{}v}\PY{p}{)}\PY{p}{:}
                         \PY{c}{\PYZsh{} forward Euler}
                         \PY{k}{if} \PY{p}{(}\PY{n+nb+bp}{self}\PY{o}{.}\PY{n}{method} \PY{o}{==} \PY{l+s}{'}\PY{l+s}{forward}\PY{l+s}{'}\PY{p}{)}\PY{p}{:}
                             
                             \PY{n}{v\PYZus{}v}\PY{p}{[}\PY{n}{i}\PY{p}{]} \PY{o}{=}\PY{p}{(}\PY{p}{(}\PY{l+m+mi}{1} \PY{o}{-} \PY{n+nb+bp}{self}\PY{o}{.}\PY{n}{dt} \PY{o}{/} \PY{n+nb+bp}{self}\PY{o}{.}\PY{n}{tau}\PY{p}{)} \PY{o}{*} \PY{n}{v\PYZus{}v}\PY{p}{[}\PY{n}{i}\PY{o}{-}\PY{l+m+mi}{1}\PY{p}{]} \PY{p}{)}\PY{o}{+} \PY{p}{(}\PY{n+nb+bp}{self}\PY{o}{.}\PY{n}{dt} \PY{o}{*} \PY{n}{i\PYZus{}v\PYZus{}tot}\PY{p}{[}\PY{n}{i}\PY{o}{-}\PY{l+m+mi}{1}\PY{p}{]} \PY{o}{/} \PY{n+nb+bp}{self}\PY{o}{.}\PY{n}{C}\PY{p}{)}
         
                         \PY{c}{\PYZsh{} backward Euler}
                         \PY{k}{elif} \PY{p}{(}\PY{n+nb+bp}{self}\PY{o}{.}\PY{n}{method} \PY{o}{==} \PY{l+s}{'}\PY{l+s}{backward}\PY{l+s}{'}\PY{p}{)}\PY{p}{:}
                             
                             \PY{n}{v\PYZus{}v}\PY{p}{[}\PY{n}{i}\PY{p}{]} \PY{o}{=} \PY{p}{(}\PY{n}{m1\PYZus{}dt\PYZus{}tau} \PY{o}{*} \PY{n}{v\PYZus{}v}\PY{p}{[}\PY{n}{i}\PY{o}{-}\PY{l+m+mi}{1}\PY{p}{]} \PY{p}{)}\PY{o}{+} \PY{p}{(}\PY{n+nb+bp}{self}\PY{o}{.}\PY{n}{dt} \PY{o}{*} \PY{p}{(}\PY{n}{i\PYZus{}v\PYZus{}tot}\PY{p}{[}\PY{n}{i}\PY{p}{]} \PY{o}{/} \PY{n+nb+bp}{self}\PY{o}{.}\PY{n}{C}\PY{p}{)}\PY{p}{)}
         
                         \PY{k}{if} \PY{p}{(}\PY{n}{v\PYZus{}v}\PY{p}{[}\PY{n}{i}\PY{p}{]} \PY{o}{>}\PY{o}{=} \PY{n+nb+bp}{self}\PY{o}{.}\PY{n}{v\PYZus{}thres}\PY{p}{)}\PY{p}{:}
                                 
                             \PY{n}{v\PYZus{}v}\PY{p}{[}\PY{n}{i}\PY{p}{]} \PY{o}{=} \PY{l+m+mf}{0.}
                             \PY{n}{s\PYZus{}v}\PY{p}{[}\PY{n}{i}\PY{p}{]} \PY{o}{=} \PY{l+m+mf}{1.}
         
                 \PY{n}{vteo} \PY{o}{=} \PY{n+nb+bp}{self}\PY{o}{.}\PY{n}{I\PYZus{}max}\PY{o}{*}\PY{n+nb+bp}{self}\PY{o}{.}\PY{n}{R}\PY{o}{*}\PY{p}{(}\PY{l+m+mi}{1} \PY{o}{-} \PY{n}{np}\PY{o}{.}\PY{n}{exp}\PY{p}{(}\PY{o}{-}\PY{n}{t\PYZus{}v} \PY{o}{/} \PY{n+nb+bp}{self}\PY{o}{.}\PY{n}{tau}\PY{p}{)}\PY{p}{)}
         
                 \PY{k}{return} \PY{n}{t\PYZus{}v}\PY{p}{,} \PY{n}{v\PYZus{}v}\PY{p}{,} \PY{n}{i\PYZus{}v\PYZus{}i}\PY{p}{,} \PY{n}{vteo}
             
             \PY{k}{def} \PY{n+nf}{freq}\PY{p}{(}\PY{n+nb+bp}{self}\PY{p}{,} \PY{n}{t\PYZus{}ref}\PY{p}{)}\PY{p}{:}
                 
                 \PY{n+nb+bp}{self}\PY{o}{.}\PY{n}{t\PYZus{}ref} \PY{o}{=} \PY{n}{t\PYZus{}ref}
                 \PY{c}{\PYZsh{} Corriente minima }
                 \PY{n}{I\PYZus{}min} \PY{o}{=} \PY{n+nb+bp}{self}\PY{o}{.}\PY{n}{v\PYZus{}thres} \PY{o}{/} \PY{n+nb+bp}{self}\PY{o}{.}\PY{n}{R}
                 
                 \PY{n}{I} \PY{o}{=} \PY{n}{np}\PY{o}{.}\PY{n}{linspace}\PY{p}{(}\PY{n}{I\PYZus{}min}\PY{p}{,}\PY{l+m+mi}{20}\PY{p}{)}
                 
                 \PY{n}{f} \PY{o}{=} \PY{l+m+mi}{1} \PY{o}{/} \PY{p}{(}\PY{n+nb+bp}{self}\PY{o}{.}\PY{n}{t\PYZus{}ref} \PY{o}{-} \PY{n+nb+bp}{self}\PY{o}{.}\PY{n}{tau} \PY{o}{*} \PY{n}{np}\PY{o}{.}\PY{n}{log}\PY{p}{(}\PY{l+m+mi}{1} \PY{o}{-} \PY{n+nb+bp}{self}\PY{o}{.}\PY{n}{v\PYZus{}thres}\PY{o}{/} \PY{p}{(}\PY{n}{I} \PY{o}{*} \PY{n+nb+bp}{self}\PY{o}{.}\PY{n}{R}\PY{p}{)}\PY{p}{)}\PY{p}{)}
                 
                 \PY{k}{return} \PY{n}{I}\PY{p}{,} \PY{n}{f}
             
             \PY{c}{\PYZsh{} Metodo para imprimir los objetos  }
             \PY{k}{def} \PY{n+nf}{\PYZus{}\PYZus{}str\PYZus{}\PYZus{}}\PY{p}{(}\PY{n+nb+bp}{self}\PY{p}{)}\PY{p}{:}  
                 
                 \PY{k}{return} \PY{l+s}{'}\PY{l+s}{[NeuronaIF: tau:}\PY{l+s+si}{\PYZpc{}s}\PY{l+s}{ R:}\PY{l+s+si}{\PYZpc{}s}\PY{l+s}{ v\PYZus{}thres:}\PY{l+s+si}{\PYZpc{}s}\PY{l+s}{]}\PY{l+s}{'} \PY{o}{\PYZpc{}} \PY{p}{(}\PY{n+nb+bp}{self}\PY{o}{.}\PY{n}{tau}\PY{p}{,} \PY{n+nb+bp}{self}\PY{o}{.}\PY{n}{R}\PY{p}{,} \PY{n+nb+bp}{self}\PY{o}{.}\PY{n}{v\PYZus{}thres}\PY{p}{)}
             
                 
\end{Verbatim}

    \begin{Verbatim}[commandchars=\\\{\}]
{\color{incolor}In [{\color{incolor}30}]:} \PY{c}{\PYZsh{}Objeto de la clase NeuronaIF}
         \PY{n}{neurona\PYZus{}1} \PY{o}{=} \PY{n}{NeuronaIF}\PY{p}{(}\PY{l+m+mf}{20.}\PY{p}{,} \PY{l+m+mf}{10.}\PY{p}{,} \PY{l+m+mf}{40.}\PY{p}{)}
\end{Verbatim}

    \begin{Verbatim}[commandchars=\\\{\}]
{\color{incolor}In [{\color{incolor}31}]:} \PY{c}{\PYZsh{}Simulamos la neurona con los parametros default}
         \PY{n}{t\PYZus{}v}\PY{p}{,} \PY{n}{v\PYZus{}v}\PY{p}{,} \PY{n}{i\PYZus{}v\PYZus{}i}\PY{p}{,} \PY{n}{vteo} \PY{o}{=} \PY{n}{neurona\PYZus{}1}\PY{o}{.}\PY{n}{simulate}\PY{p}{(}\PY{p}{)}
\end{Verbatim}

    \begin{Verbatim}[commandchars=\\\{\}]
{\color{incolor}In [{\color{incolor}32}]:} \PY{k}{print} \PY{n}{neurona\PYZus{}1}
\end{Verbatim}

    \begin{Verbatim}[commandchars=\\\{\}]
[NeuronaIF: tau:20.0 R:10.0 v\_thres:40.0]
    \end{Verbatim}

    \begin{Verbatim}[commandchars=\\\{\}]
{\color{incolor}In [{\color{incolor}33}]:} \PY{o}{\PYZpc{}}\PY{k}{matplotlib} \PY{n}{inline}
\end{Verbatim}

    \begin{Verbatim}[commandchars=\\\{\}]
{\color{incolor}In [{\color{incolor}34}]:} \PY{k+kn}{import} \PY{n+nn}{matplotlib.pyplot} \PY{k+kn}{as} \PY{n+nn}{plt}
         \PY{n}{plt}\PY{o}{.}\PY{n}{rcParams}\PY{p}{[}\PY{l+s}{'}\PY{l+s}{figure.figsize}\PY{l+s}{'}\PY{p}{]} \PY{o}{=} \PY{l+m+mi}{8}\PY{p}{,}\PY{l+m+mi}{6}
\end{Verbatim}

    \begin{Verbatim}[commandchars=\\\{\}]
{\color{incolor}In [{\color{incolor}35}]:} \PY{n}{fig}\PY{p}{,} \PY{n}{axes} \PY{o}{=} \PY{n}{plt}\PY{o}{.}\PY{n}{subplots}\PY{p}{(}\PY{p}{)}
         \PY{n}{axes}\PY{o}{.}\PY{n}{plot}\PY{p}{(}\PY{n}{t\PYZus{}v}\PY{p}{,} \PY{n}{v\PYZus{}v}\PY{p}{,}\PY{l+s}{'}\PY{l+s}{b}\PY{l+s}{'}\PY{p}{)}
         \PY{n}{axes}\PY{o}{.}\PY{n}{plot}\PY{p}{(}\PY{n}{t\PYZus{}v}\PY{p}{,} \PY{n}{i\PYZus{}v\PYZus{}i}\PY{p}{,} \PY{l+s}{'}\PY{l+s}{r}\PY{l+s}{'}\PY{p}{)}
         \PY{n}{axes}\PY{o}{.}\PY{n}{plot}\PY{p}{(}\PY{n}{t\PYZus{}v}\PY{p}{,} \PY{n}{vteo}\PY{p}{,} \PY{l+s}{'}\PY{l+s}{k}\PY{l+s}{'}\PY{p}{)}
         
         \PY{n}{axes}\PY{o}{.}\PY{n}{set\PYZus{}xlabel}\PY{p}{(}\PY{l+s}{'}\PY{l+s}{\PYZdl{}t\PYZdl{}}\PY{l+s}{'}\PY{p}{,} \PY{n}{fontsize}\PY{o}{=}\PY{l+m+mi}{20}\PY{p}{)}
         \PY{n}{axes}\PY{o}{.}\PY{n}{set\PYZus{}title}\PY{p}{(}\PY{l+s}{'}\PY{l+s}{Simulacion de una Neurona IF}\PY{l+s}{'}\PY{p}{,} \PY{n}{fontsize}\PY{o}{=}\PY{l+m+mi}{20}\PY{p}{)}
         \PY{n}{fig}\PY{o}{.}\PY{n}{savefig}\PY{p}{(}\PY{l+s}{'}\PY{l+s}{plot1.pdf}\PY{l+s}{'}\PY{p}{)}
\end{Verbatim}

   
\begin{figure}[h!]
   \centering
   \includegraphics[width=0.70\textwidth]{plot1.eps}
   \caption{ } \label{fig:simu1}     
\end{figure}
    \subsection{b)}

    De acuerdo a la ecuación de discretización del método \emph{forward} si
llamamos $a = 1-\frac{dt}{\tau}$ y evaluamos vemos que se puede
generalizar la formula a la siguiente expresión:

\[V^{j}=a^{j-1} V^{1}+dt\sum_{i=1}^{j-1}a^{j-i-1}f^{i}\]

Al ser una serie geométrica $|a|< 1$ y además para asegurar convergencia
$dt < 2\tau$

Vamos a simular nuestra neurona con un $dt = 41$ que no cumpliría la
condición de convergencia

    \begin{Verbatim}[commandchars=\\\{\}]
{\color{incolor}In [{\color{incolor}36}]:} \PY{c}{\PYZsh{} Simulacion de una neurona con un dt=41}
         \PY{n}{t\PYZus{}v}\PY{p}{,} \PY{n}{v\PYZus{}v}\PY{p}{,} \PY{n}{i\PYZus{}v\PYZus{}i}\PY{p}{,} \PY{n}{vteo} \PY{o}{=} \PY{n}{neurona\PYZus{}1}\PY{o}{.}\PY{n}{simulate}\PY{p}{(}\PY{l+m+mi}{41}\PY{p}{)}
\end{Verbatim}

    \begin{Verbatim}[commandchars=\\\{\}]
{\color{incolor}In [{\color{incolor}37}]:} \PY{n}{fig}\PY{p}{,} \PY{n}{axes} \PY{o}{=} \PY{n}{plt}\PY{o}{.}\PY{n}{subplots}\PY{p}{(}\PY{p}{)}
         \PY{n}{axes}\PY{o}{.}\PY{n}{plot}\PY{p}{(}\PY{n}{t\PYZus{}v}\PY{p}{,} \PY{n}{v\PYZus{}v}\PY{p}{,}\PY{l+s}{'}\PY{l+s}{b}\PY{l+s}{'}\PY{p}{)}
         \PY{n}{axes}\PY{o}{.}\PY{n}{plot}\PY{p}{(}\PY{n}{t\PYZus{}v}\PY{p}{,} \PY{n}{i\PYZus{}v\PYZus{}i}\PY{p}{,} \PY{l+s}{'}\PY{l+s}{r}\PY{l+s}{'}\PY{p}{)}
         \PY{n}{axes}\PY{o}{.}\PY{n}{plot}\PY{p}{(}\PY{n}{t\PYZus{}v}\PY{p}{,} \PY{n}{vteo}\PY{p}{,} \PY{l+s}{'}\PY{l+s}{k}\PY{l+s}{'}\PY{p}{)}
         
         \PY{n}{axes}\PY{o}{.}\PY{n}{set\PYZus{}xlabel}\PY{p}{(}\PY{l+s}{'}\PY{l+s}{\PYZdl{}t\PYZdl{}}\PY{l+s}{'}\PY{p}{,} \PY{n}{fontsize}\PY{o}{=}\PY{l+m+mi}{20}\PY{p}{)}
         \PY{n}{axes}\PY{o}{.}\PY{n}{set\PYZus{}title}\PY{p}{(}\PY{l+s}{'}\PY{l+s}{Simulacion de una Neurona IF}\PY{l+s}{'}\PY{p}{,} \PY{n}{fontsize}\PY{o}{=}\PY{l+m+mi}{20}\PY{p}{)}
\end{Verbatim}

            \begin{Verbatim}[commandchars=\\\{\}]
{\color{outcolor}Out[{\color{outcolor}37}]:} <matplotlib.text.Text at 0x346d790>
\end{Verbatim}
        
   \begin{figure}[h!]
   \centering
   \includegraphics[width=0.70\textwidth]{plot2.eps}
   \caption{ } \label{fig:simu1}     
\end{figure}
   
    
    \subsection{c) - d)}

    Como sabemos la solución a a la ecuación diferencial es:

\[v(t)=IR(1-exp(\frac{-t}{\tau}))\]

cuando $v(t)=v_{thres}$ tenemos:

\[v_{thres}=IR(1-exp(\frac{-t_{thres}}{\tau}))\]

Por ello si despejamos $t_{thres}$ nos queda:

\[t_{thres}=-\tau \, log(\frac{1-v_{thres}}{IR})\]

Luego como la frecuencia es la inversa del tiempo y además teniendo en
cuenta que existe un tiempo muerto($t_{ref}$) donde la neurona no puede
ser exitada

\[f=\frac{1}{t_{ref}+t_{thres}}=\frac{1}{t_{ref}-\tau \, log(1-\frac{v_{thres}}{IR})}\]

    \begin{Verbatim}[commandchars=\\\{\}]
{\color{incolor}In [{\color{incolor}38}]:} \PY{n}{t\PYZus{}v}\PY{p}{,} \PY{n}{v\PYZus{}v}\PY{p}{,} \PY{n}{i\PYZus{}v\PYZus{}i}\PY{p}{,} \PY{n}{vteo} \PY{o}{=} \PY{n}{neurona\PYZus{}1}\PY{o}{.}\PY{n}{simulate}\PY{p}{(}\PY{l+m+mi}{1}\PY{p}{)}
\end{Verbatim}

    \begin{Verbatim}[commandchars=\\\{\}]
{\color{incolor}In [{\color{incolor}39}]:} \PY{n}{I}\PY{p}{,}\PY{n}{f} \PY{o}{=} \PY{n}{neurona\PYZus{}1}\PY{o}{.}\PY{n}{freq}\PY{p}{(}\PY{o}{.}\PY{l+m+mo}{01}\PY{p}{)}  \PY{c}{\PYZsh{} el parametro que utilizamos es t\PYZus{}ref = 0.1 ms}
\end{Verbatim}

    \begin{Verbatim}[commandchars=\\\{\}]
{\color{incolor}In [{\color{incolor}40}]:} \PY{n}{f}
\end{Verbatim}

            \begin{Verbatim}[commandchars=\\\{\}]
{\color{outcolor}Out[{\color{outcolor}40}]:} array([ 0.        ,  0.01934612,  0.02545683,  0.03074147,  0.03565575,
                 0.04035922,  0.04492744,  0.04940219,  0.05380896,  0.0581644 ,
                 0.06247993,  0.06676371,  0.07102172,  0.0752585 ,  0.07947753,
                 0.08368154,  0.08787273,  0.09205285,  0.09622334,  0.1003854 ,
                 0.10454001,  0.10868801,  0.11283011,  0.11696691,  0.12109893,
                 0.12522662,  0.12935037,  0.13347051,  0.13758735,  0.14170115,
                 0.14581214,  0.14992053,  0.1540265 ,  0.15813021,  0.16223182,
                 0.16633146,  0.17042924,  0.17452528,  0.17861966,  0.1827125 ,
                 0.18680385,  0.1908938 ,  0.19498242,  0.19906977,  0.2031559 ,
                 0.20724087,  0.21132473,  0.21540752,  0.21948928,  0.22357005])
\end{Verbatim}
        
    \begin{Verbatim}[commandchars=\\\{\}]
{\color{incolor}In [{\color{incolor}41}]:} \PY{n}{I}
\end{Verbatim}

            \begin{Verbatim}[commandchars=\\\{\}]
{\color{outcolor}Out[{\color{outcolor}41}]:} array([  4.        ,   4.32653061,   4.65306122,   4.97959184,
                  5.30612245,   5.63265306,   5.95918367,   6.28571429,
                  6.6122449 ,   6.93877551,   7.26530612,   7.59183673,
                  7.91836735,   8.24489796,   8.57142857,   8.89795918,
                  9.2244898 ,   9.55102041,   9.87755102,  10.20408163,
                 10.53061224,  10.85714286,  11.18367347,  11.51020408,
                 11.83673469,  12.16326531,  12.48979592,  12.81632653,
                 13.14285714,  13.46938776,  13.79591837,  14.12244898,
                 14.44897959,  14.7755102 ,  15.10204082,  15.42857143,
                 15.75510204,  16.08163265,  16.40816327,  16.73469388,
                 17.06122449,  17.3877551 ,  17.71428571,  18.04081633,
                 18.36734694,  18.69387755,  19.02040816,  19.34693878,
                 19.67346939,  20.        ])
\end{Verbatim}
        
    \begin{Verbatim}[commandchars=\\\{\}]
{\color{incolor}In [{\color{incolor}44}]:} \PY{c}{\PYZsh{} Plots de las frecuencias de spiking}
         \PY{n}{plt}\PY{o}{.}\PY{n}{plot}\PY{p}{(}\PY{n}{I}\PY{p}{,}\PY{n}{f}\PY{p}{,}\PY{l+s}{'}\PY{l+s}{o}\PY{l+s}{'}\PY{p}{)}
         \PY{n}{plt}\PY{o}{.}\PY{n}{xlabel}\PY{p}{(}\PY{l+s}{'}\PY{l+s}{I[nA]}\PY{l+s}{'}\PY{p}{,} \PY{n}{fontsize}\PY{o}{=}\PY{l+m+mi}{20}\PY{p}{)}
         \PY{n}{plt}\PY{o}{.}\PY{n}{ylabel}\PY{p}{(}\PY{l+s}{'}\PY{l+s}{frequency}\PY{l+s}{'}\PY{p}{,} \PY{n}{fontsize}\PY{o}{=}\PY{l+m+mi}{20}\PY{p}{)}
         \PY{n}{plt}\PY{o}{.}\PY{n}{title}\PY{p}{(}\PY{l+s}{'}\PY{l+s}{Curva teorica de frecuencias}\PY{l+s}{'}\PY{p}{,} \PY{n}{fontsize}\PY{o}{=}\PY{l+m+mi}{20}\PY{p}{)}
\end{Verbatim}

            \begin{Verbatim}[commandchars=\\\{\}]
{\color{outcolor}Out[{\color{outcolor}44}]:} <matplotlib.text.Text at 0x3ba5b10>
\end{Verbatim}
        
    \begin{figure}[h!]
   \centering
   \includegraphics[width=0.70\textwidth]{plot3.eps}
   \caption{ } \label{fig:simu1}     
\end{figure}
   
    
    \begin{Verbatim}[commandchars=\\\{\}]
{\color{incolor}In [{\color{incolor}}]:} 
\end{Verbatim}


    % Add a bibliography block to the postdoc
    
    
    
    \end{document}
